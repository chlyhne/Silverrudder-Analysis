\documentclass[11pt]{article}
\usepackage[a4paper,margin=2.5cm]{geometry}
\usepackage{graphicx}
\usepackage{hyperref}
\usepackage{float}
\title{Silverrudder 2025 Regatta Summary}
\author{}
\date{\today}
\begin{document}
\maketitle
\section*{Overview}
This document summarizes pace deltas (minutes per NM) by leg.
\paragraph{Purpose.}
The purpose of this report is to give skippers a practical tool for evaluating and analyzing their performance in Silverrudder 2025. By comparing each boat to a common baseline at the same \emph{place} on the course (progress along the rhumb line), the plots highlight where time was gained or lost on each leg and where performance was stable versus inconsistent.
\paragraph{How to read.}
Each leg figure shows, for every boat, the distribution of per-sample \emph{pace delta} relative to the fleet mean pace at the same \emph{progress} along the rhumb line (average route). Negative values mean a boat was faster than the baseline (lower minutes per NM), while positive values mean slower. The box spans the interquartile range (25--75\%), the horizontal line is the median, and the dot is the mean; whiskers show the full min/max observed within that leg.
\paragraph{Definitions (speed, pace, and mean pace).}
The input data contains a \emph{speed} sample for each GPS record (after unit conversion to knots, $\mathrm{kn}$).
\emph{Pace} is the inverse representation: minutes per nautical mile. For each sample,
\[
\mathrm{pace}\,[\mathrm{min}/\mathrm{NM}] = \frac{60}{v\,[\mathrm{kn}]}.
\]
For example, $6.0\,\mathrm{kn} \Rightarrow 10.00\,\mathrm{min}/\mathrm{NM}$ and $5.0\,\mathrm{kn} \Rightarrow 12.00\,\mathrm{min}/\mathrm{NM}$.
\paragraph{Why pace (not speed).}
Pace is directly ``time per distance'', which makes it easier to reason about where time is gained or lost along the course. Because pace is the inverse of speed,
\[
\mathrm{pace}=\frac{60}{v},
\]
the relationship is nonlinear: the same speed change costs more minutes per nautical mile when the boat is already slow. Example: $6\,\mathrm{kn}\rightarrow 5\,\mathrm{kn}$ changes pace from $10.00$ to $12.00\,\mathrm{min}/\mathrm{NM}$ (+2.00), while $3\,\mathrm{kn}\rightarrow 2\,\mathrm{kn}$ changes pace from $20.00$ to $30.00\,\mathrm{min}/\mathrm{NM}$ (+10.00). Those extra minutes per mile add up directly into finish-time differences. Expressing performance as pace deltas highlights the locations where time is being lost, even when boats pass those locations at different times.
\emph{Mean pace} in this report refers to a \emph{baseline} computed at a given location along the course: all boat samples are first projected to the rhumb line (average route) to obtain progress, then within a local progress window the per-sample paces are averaged to form a mean pace curve versus progress (and lightly smoothed along progress). Note that ``mean pace'' is \emph{not} the same as converting the mean speed, because inversion is nonlinear: in general $\overline{(60/v)} \neq 60/\overline{v}$.
\paragraph{Pace delta.}
For each boat sample, the plotted value is
\[
\Delta \mathrm{pace} = \mathrm{pace}_{\mathrm{boat}} - \mathrm{pace}_{\mathrm{mean}}(\mathrm{progress}).
\]
Negative $\Delta\mathrm{pace}$ means faster-than-baseline at that point on the course (lower minutes per NM), while positive means slower-than-baseline.
\paragraph{High-level observations.}
Use the plot shapes as direct diagnostics:
\begin{itemize}
\item \textbf{Sign:} a mean dot or median line \emph{below} $0$ means faster-than-baseline at the same place on the course; \emph{above} $0$ means slower-than-baseline.
\item \textbf{Consistency:} a short box (small IQR) means the boat stayed close to a single pace delta through the leg; a tall box means the pace delta moved around a lot within the leg.
\item \textbf{Time leaks:} a long upper whisker (large positive deltas) means the boat had patches where it was dramatically slower than the baseline at specific locations; those patches dominate time lost on that leg.
\end{itemize}
\clearpage
\section*{Leg 1: Start/Finish-Thuro}
\begin{figure}[H]
\centering
\includegraphics[width=\linewidth]{figures/pace-delta-leg-01-start-finish-thuro.pdf}
\caption{Pace delta by boat for leg 1.}
\end{figure}
\paragraph{What stands out.}
This leg contains a large start-phase effect. The wide spread is driven by late starters and congestion: when a boat is still stuck in the start pack (very low speed), its pace increases sharply and produces large positive $\Delta\mathrm{pace}$ outliers. Boats that clear the start line and traffic quickly produce distributions closer to $0$ (or below), while boats delayed in congestion show long positive tails that mostly represent start delay rather than steady sailing performance.
\clearpage
\section*{Leg 2: Thuro-Knudshoved}
\begin{figure}[H]
\centering
\includegraphics[width=\linewidth]{figures/pace-delta-leg-02-thuro-knudshoved.pdf}
\caption{Pace delta by boat for leg 2.}
\end{figure}
\paragraph{What stands out.}
Compared to leg 1, the distributions are tighter: many boats sit closer to $0$ and the boxes are shorter. Boats that remain clearly offset from $0$ through this leg are carrying a persistent advantage or disadvantage at the same places on the course (not a single isolated incident).
\clearpage
\section*{Leg 3: Knudshoved-Fynshoved}
\begin{figure}[H]
\centering
\includegraphics[width=\linewidth]{figures/pace-delta-leg-03-knudshoved-fynshoved.pdf}
\caption{Pace delta by boat for leg 3.}
\end{figure}
\paragraph{What stands out.}
This is a compression leg: many boats sit close to $0$ and the main differentiator is consistency. A narrow box means the boat stayed close to its typical pace delta through the leg; a long positive whisker means the boat had brief but very costly slowdowns at specific places.
\clearpage
\section*{Leg 4: Fynshoved-Aebelo}
\begin{figure}[H]
\centering
\includegraphics[width=\linewidth]{figures/pace-delta-leg-04-fynshoved-aebelo.pdf}
\caption{Pace delta by boat for leg 4.}
\end{figure}
\paragraph{What stands out.}
The slow side shows the important structure: when a boat's box lies mostly above $0$, the boat is slower than the baseline through most of the leg. When the upper whisker extends far above the box, the boat has a small number of locations where it loses a large amount of time per mile relative to the baseline.
\clearpage
\section*{Leg 5: Aebelo-Strib}
\begin{figure}[H]
\centering
\includegraphics[width=\linewidth]{figures/pace-delta-leg-05-aebelo-strib.pdf}
\caption{Pace delta by boat for leg 5.}
\end{figure}
\paragraph{What stands out.}
This leg shows clearer separation: boats spread further away from $0$ and the slow side shows larger spreads for some boats. When a boat has a high median \emph{and} a tall box, the boat is slower than baseline most of the time and also inconsistent within the leg; that combination produces large cumulative time loss.
\clearpage
\section*{Leg 6: Strib-Hindsgavl}
\begin{figure}[H]
\centering
\includegraphics[width=\linewidth]{figures/pace-delta-leg-06-strib-hindsgavl.pdf}
\caption{Pace delta by boat for leg 6.}
\end{figure}
\paragraph{What stands out.}
This leg separates boats into a faster group (negative deltas) and a slower group (positive deltas). Within each group, compare spreads: two boats have similar medians, but the boat with the narrower box is the boat with fewer large deviations from its typical pace delta through the leg.
\clearpage
\section*{Leg 7: Hindsgavl-Helnaes}
\begin{figure}[H]
\centering
\includegraphics[width=\linewidth]{figures/pace-delta-leg-07-hindsgavl-helnaes.pdf}
\caption{Pace delta by boat for leg 7.}
\end{figure}
\paragraph{What stands out.}
This leg makes it easy to spot both speed and reliability. Boats with boxes centered below $0$ are faster-than-baseline; boats with boxes centered above $0$ are slower-than-baseline. A tight box with one long positive whisker points to a specific slow patch: most of the leg was fine, but one location produced a large time loss.
\clearpage
\section*{Leg 8: Helnaes-Svelmo}
\begin{figure}[H]
\centering
\includegraphics[width=\linewidth]{figures/pace-delta-leg-08-helnaes-svelmo.pdf}
\caption{Pace delta by boat for leg 8.}
\end{figure}
\paragraph{What stands out.}
This leg shows the same two ingredients: offset from $0$ (faster or slower) and spread (stable or variable). Boats that combine a negative center and a narrow box are fast \emph{and} steady; boats that combine a positive center and a tall box are slow \emph{and} inconsistent.
\clearpage
\section*{Leg 9: Svelmo-Start/Finish}
\begin{figure}[H]
\centering
\includegraphics[width=\linewidth]{figures/pace-delta-leg-09-svelmo-start-finish.pdf}
\caption{Pace delta by boat for leg 9.}
\end{figure}
\paragraph{What stands out.}
This final leg contains the largest slow outliers in the set of plots: some boats have long positive tails. Those tails correspond to locations where the boat spent substantial time at low speed, which drives pace upward (minutes per mile). Boats with a strong slow tail in this leg have a small set of places that dominate their total time loss on the run back to Start/Finish.
\clearpage
\section*{Pace delta by boat (all legs)}
% Auto-generated by silver.py. Do not edit by hand.
\clearpage
\subsection*{Boat: Saga}
\begin{figure}[H]
\centering
\includegraphics{figures/\figurevariant/pace-delta-boat-saga.pdf}
\caption{Pace delta by leg for Saga.}
\end{figure}
\clearpage
\subsection*{Boat: Sweet Escape}
\begin{figure}[H]
\centering
\includegraphics{figures/\figurevariant/pace-delta-boat-sweet-escape.pdf}
\caption{Pace delta by leg for Sweet Escape.}
\end{figure}
\clearpage
\subsection*{Boat: Mercutio}
\begin{figure}[H]
\centering
\includegraphics{figures/\figurevariant/pace-delta-boat-mercutio.pdf}
\caption{Pace delta by leg for Mercutio.}
\end{figure}
\clearpage
\subsection*{Boat: Gladys}
\begin{figure}[H]
\centering
\includegraphics{figures/\figurevariant/pace-delta-boat-gladys.pdf}
\caption{Pace delta by leg for Gladys.}
\end{figure}
\clearpage
\subsection*{Boat: Aegir 2.0}
\begin{figure}[H]
\centering
\includegraphics{figures/\figurevariant/pace-delta-boat-aegir-2-0.pdf}
\caption{Pace delta by leg for Aegir 2.0.}
\end{figure}
\clearpage
\subsection*{Boat: Nordri}
\begin{figure}[H]
\centering
\includegraphics{figures/\figurevariant/pace-delta-boat-nordri.pdf}
\caption{Pace delta by leg for Nordri.}
\end{figure}
\clearpage
\subsection*{Boat: Vera}
\begin{figure}[H]
\centering
\includegraphics{figures/\figurevariant/pace-delta-boat-vera.pdf}
\caption{Pace delta by leg for Vera.}
\end{figure}
\clearpage
\subsection*{Boat: Moana}
\begin{figure}[H]
\centering
\includegraphics{figures/\figurevariant/pace-delta-boat-moana.pdf}
\caption{Pace delta by leg for Moana.}
\end{figure}
\clearpage
\subsection*{Boat: My}
\begin{figure}[H]
\centering
\includegraphics{figures/\figurevariant/pace-delta-boat-my.pdf}
\caption{Pace delta by leg for My.}
\end{figure}
\clearpage
\subsection*{Boat: Julia}
\begin{figure}[H]
\centering
\includegraphics{figures/\figurevariant/pace-delta-boat-julia.pdf}
\caption{Pace delta by leg for Julia.}
\end{figure}
\clearpage
\subsection*{Boat: Lets Sea}
\begin{figure}[H]
\centering
\includegraphics{figures/\figurevariant/pace-delta-boat-lets-sea.pdf}
\caption{Pace delta by leg for Lets Sea.}
\end{figure}
\clearpage
\subsection*{Boat: Chaya}
\begin{figure}[H]
\centering
\includegraphics{figures/\figurevariant/pace-delta-boat-chaya.pdf}
\caption{Pace delta by leg for Chaya.}
\end{figure}
\clearpage
\subsection*{Boat: Samba}
\begin{figure}[H]
\centering
\includegraphics{figures/\figurevariant/pace-delta-boat-samba.pdf}
\caption{Pace delta by leg for Samba.}
\end{figure}

\end{document}
